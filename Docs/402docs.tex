\documentclass[11pt]{article}


\usepackage{geometry}
\geometry{letterpaper}

\usepackage{doc}
\usepackage{cite}
\usepackage[margin=1cm]{caption}
\usepackage{url}
\usepackage{graphicx}
\usepackage{epstopdf}
\DeclareGraphicsRule{.tif}{png}{.png}{`convert #1 `dirname #1`/`basename #1 .tif`.png}
\usepackage{etoolbox}
\patchcmd{\thebibliography}{\section*}{\section}{}{}

\usepackage{amsmath,etoolbox}
\numberwithin{page}{section}% Number page by section
\renewcommand{\thepage}{\thesection--\arabic{page}}% Page numbering style
\makeatletter
% Make sure that page starts from 1 with every \section
\patchcmd{\@sect}% <cmd>
  {\protected@edef}% <search>
  {\def\arg{#1}\def\arg@{section}%
   \ifx\arg\arg@\stepcounter{page}\fi%
   \protected@edef}% <replace>
  {}{}% <success><failure>
\makeatother

\title{Camera Capture for Maya}
\author{Cameron Billingham}
\date{January 19, 2016}


\begin{document}


\maketitle
\thispagestyle{empty}
\pagebreak

\tableofcontents
\thispagestyle{empty}
\pagebreak

\section{Preliminary Proposals}

\begin{enumerate}
\item
An application that allows Maya users to view viewport renders in real-time on their iPad or iPhone and allows them to manipulate the camera using the gyroscope and accelerometer on the device. The application will allow the user to connect the mobile device to the computer running Maya over Wi-Fi.
\item
An application that allows users to stream audio from their device to other Bluetooth devices in sync. The application allows connection to multiple Bluetooth devices including other devices with the application and stream whatever they are currently listening to to another device.
\item
A WebGL program that shows the manipulation and creation of a view frustum and projection. This tool would be used to allow students to see the render box of graphics scene to better understand the idea of a orthographic and projection matrix.
\end{enumerate}

\clearpage

\section{Proposal Document}
\subsection{General Description}
Camera Capture for Maya is an iOS application that allows a Maya user to use their iPad or iPhone a camera capture device. The application will show a live view of the current camera frame as displayed in the Maya viewport. Users will be able to record camera motion in the application by playing the animation scene in real time. The app will allow the Maya camera to be controlled by the position, rotation, and movement of the device. For example, if the user wants to pan the Maya camera to the left they can turn their device to the left to control the camera in Maya. 

The tool will consist of two seperate programs. The first will be a Maya plugin running on a desktop or laptop computer with Maya that will primarily be written in Python. This program will recieve camera translation data and also transmit the camera viewport to the device. The second will be the iOS application, written in Swift, that reads the gyroscope and accelerometer data of the device, as well as display the camera capture interface to the user. The two pieces will communicate over a network port.

\subsection{Related Work}
My application is partially inspired by an iOS application called Cameraman for Maya that allows the user to control the camera using the gyroscope on their iOS device. However, this application does not show the user what the current frame looks like on their device and it is limited to recording only the rotation of the camera using the gyroscope. This limited functionality prohibits the cinematrographer from seeing their viewport as they move the camera. My application will add this needed funcationality as well as allow the cinematographer more freedom to position the camera in any way.

\subsection{Justification}
This project was inspired by the Animation Pipeline class that I took last year. Being a film production and computer science major I am very interested in software engineering in the realm of animated films and computer graphics. I expecially enjoy working with the camera on live action film sets and I ahve always felt that working with the camera in Maya is very artificial because you cannot physically move the camera and position it with your hands. Professional animation studios have expensive equipment to capture handheld and other physical camera movement. This application will bring this ability to easily create camera movement to the average consumer without forcing them to buy very expensive equipment. 

My goal with this application was to explore my interest in software engineering for animation by building on my experiences in Animation Pipeline and Computer Graphics classes. I have previous experience working in Maya and with programming in Python, but I will also be challenging myself to learn new technologies including the Swift programming language, video frame streaming, and network communication.

\clearpage
\bibliography{402}{}
\bibliographystyle{plain}

\end{document}
